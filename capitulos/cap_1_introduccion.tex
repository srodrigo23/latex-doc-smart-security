\chapter{Introducción}

% \begin{center}
%     \textit{
%         El presente proyecto demuestra la implementación de un sistema como propuesta de solución tecnológica para la alerta inmediata durante hechos/sucesos que pongan en peligro la integridad física y material en el hogar, como ser: fuego, humo, presencia de intrusos y violencia doméstica.
%     }
% \end{center}
La seguridad es un término usado para la referencia a la ausencia de riesgo o a la confianza explicita en algo o alguien; pero este panorama toma diversos sentidos según el campo en el que se referencia la seguridad. Aunque su objetivo consista en reducir el riesgo a niveles aceptables, el mismo es inherente a cualquier actividad o situación y en ninguna circunstancia puede ser eliminado.\\

Desde la aparición del hombre sobre el territorio terreste, siempre prevaleció su instinto de supervivencia, donde surge su necesidad de obtener y/o brindar seguridad ante cualquier peligro que ponga en riesgo su integridad física y la de sus seres más cercanos. Cuando las primeras sociedades se formaron, una de las principales tareas del estado fue administrar justicia y brindar seguridad. Por estas razones surge la necesidad de obtener u ofrecer seguridad para minimizar los riesos ante cualquier peligro.\\

En el ámbito de la seguridad, la video-vigilancia llega a ser el acto de observar una escena o escenas en busca de comportamientos específicos que podrian ser anormales o podrian indicar una posible emergencia o la existencia de un comportamiento impropio \cite{NORMAN:201795}. Los sistemas de video vigilancia de la actualidad se han convertido en una herramienta esencial de la seguridad para mantener ``observado'' un espacio muy importante para el que requiere el sistema; donde el mismo esta compuesto por un conjunto de cámaras, monitores y grabadoras donde estos elementos forman parte esencial del sistema. Estos sistemas pueden ser instalados tanto en interiores como en exteriores de una propiedad o establecimiento especialmente en lugares que se desea mantener una vigilancia constante.\\

La tecnología actual ha permitido automatizar la mayoría de las tareas que los humanos realizan y la video vigilancia no es la excepción. Con los continuos avances tecnológicos cada vez se desarrollan sistemas más robustos y avanzados, permitiendo incrementar su eficacia y confiabilidad; por ejemplo la capacidad de poder vigilar en la oscuridad gracias a la tecnología de visión nocturna. Pero el campo más fascinante dentro de estos avances es el de la Inteligencia Artificial y específicamente la rama de la ``Visión por Computadora''. Gracias a las técnicas utilizadas en este campo de investigación una computadora con el apoyo de redes neuronales tiene la capacidad de identificar objetos, siluetas y/o elementos dentro de una escena captada por una cámara.\\

Estas nuevas capacidades pueden ser explotadas en un sin fin de actividades diarias donde necesaria la supervision del ojo humano permitiendo aún más una automatización inminente. Un rápido uso de estos avances se ven en el campo de la seguridad, específicamente en los sistemas de video vigilancia permitiendo un reconocimiento de los elementos que se encuentran en una escena de forma automática. El problema a afrontar a partir de este punto es evaluar si lo que esta siendo identificado en una escena representa un peligro para las personas.

\section{Antecedentes}
En la actualidad es común que empresas e instituciones tengan instalados sistemas de seguridad en sus ambientes como ser: oficinas, sitios de producción, almacenes, entradas, recepción, etc. pero realmente no solo las empresas tienen algún riesgo de situación de peligro o robo, si no también las personas en sus respectivos hogares.\\

Con el contínuo crecimiento del mercado de la seguridad, el precio de los equipos de video vigilancia tendieron a decrecer. Este hecho asociado con el incremento de la inseguridad independientemente de cada país, promueve los siguientes escenarios: un incremento en el uso de sistemas de video vigilancia, sistemas con varias cámaras funcionando al mismo tiempo siendo monitorieadas solo por un usuario el cual no esta disponible todo el tiempo y la no capacidad de estos sistemas en el reconocimiento de  elementos en una escena para su análisis de forma automática.\

\section{Descripción del Problema}
Cuando el responsable de una casa esta ausente, nadie esta vigilando su hogar de manera que la preocupación de que esté todo normal en su hogar esta presente. Si en el peor de los casos llegase a ocurrir algo en su hogar, esta persona solo se enteraria si algún vecino se comunica con él para avisarle lo sucedido o enterearse directamente a su regreso. Un sistema de video vigilancia con la capacidad de identificar movimiento y situaciones de peligro como ser: presencia de intrusos, fuego y humo; podria disminuir los daños efectuados por las situaciones descritas por medio de una acción inmediata por parte del usuario a partir de una notificaión inmediata y una visualización en tiempo real de lo que estan captando las cámaras.\

\subsection{Definición del problema}
Dificultad para advertir de forma inmediata situaciones de peligro en el hogar.

\section{Objetivos del Proyecto}
A continuación se presentan el objetivo general y los objetivos específicos.

\subsection{Objetivo General}
Facilitar la alerta inmediata ante situaciones de peligro en el hogar por medio de un sistema de video-vigilancia inteligente.

\subsection{Objetivos Específicos}
\begin{enumerate}
    \item Describir todos los factores que implican el proceso de transmisión de datos por la red.
    \item Especificar el proceso de análisis y procesamiento de imágenes con inteligencia artificial.
    \item Proveer una red neuronal para el reconocimiento y análisis de video.
    \item Identificar las partes que conforman el proceso de transmisión de video.
    \item Describir medios para la interacción entre la transmisión y el análisis de imágenes.
    \item Proveer el medio de acceso y notificación entre el sistema y el usuario.
\end{enumerate}

\section{Justificación}
El riesgo de que un suceso ponga en peligro la integridad fisica y material de las personas esta presente cada día y en cualquier lugar. A pesar de que esta posibilidad es imposible de eliminar, se puede buscar mecanismos para poder contrarrestar el impacto que pueden ocasionar dichos sucesos donde queremos evitarlos. Las situaciones más comunes que representan un peligro para la integridad física y material del hogar son: la presencia de intrusos en ausencia del encargado del hogar y la presencia de fuego y/o humo en el interior y/o exterior del hogar.\\

Los sistemas de video vigilancia permiten visualizar en tiempo real lo que las cámaras estan captando, pero se necesita de una persona que revise constantemente dicha transmisión para poder identificar y alertar sobre las situaciones que se acaban de describir. Si la cantidad de cámaras es considerable la eficacia del operador del sistema disminuye al tener que revisar la transmisión de varias cámaras. Aprovechando la tecnología actual se plantea la implementación de un prototipo de sistema de video vigilancia inteligente que permita retransmitir de manera remota lo que estan captando lás cámaras, alertando al usuario sobre los sucesos antes descritos, despues de ser identificados por medio de técnicas de visión por computadora y redes neuronales, para poder actuar disminuyendo el impacto de estos sucesos en el hogar.

\section{Alcances y límites}
\begin{itemize}
    \item El servicio de transmisión en vivo será implementado en un servidor en línea.
    \item Se desarrollara un aplicación móvil para la notificación de alguna situación de peligro.
    \item Se identificará: fuego, humo, personas no identificadas, y acciones violentas.
    \item Será posible poder ver en vivo lo que se esta trasmitiendo desde el hogar.
    \item Se procesará el video para reconocer situaciones de peligro.
    \item Se proveerá una red neuronal para procesar imágenes.
\end{itemize}