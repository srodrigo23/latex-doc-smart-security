\chapter{Introducción}

% \begin{center}
%     \textit{
%         El presente proyecto demuestra la implementación de un sistema como propuesta de solución tecnológica para la alerta inmediata durante hechos/sucesos que pongan en peligro la integridad física y material en el hogar, como ser: fuego, humo, presencia de intrusos y violencia doméstica.
%     }
% \end{center}
El término ``Seguridad'' es usado para referir a la ausencia de riesgo o a la confianza explicita en algo o alguien; pero este panorama toma diversos sentidos según el campo en el que se enfoca la seguridad. Aunque el objetivo consista en reducir el riesgo a niveles aceptables, el mismo es inherente a cualquier actividad o situación y en ninguna circunstancia puede ser eliminado.\\

Desde la aparición del hombre sobre la faz de la tierra, siempre prevaleció el instinto de supervivencia, donde surge la necesidad de obtener y/o brindar seguridad ante cualquier peligro que ponga en riesgo la integridad física propia y la de sus seres más cercanos. Cuando las primeras sociedades se formaron, una de las principales tareas del estado fue administrar justicia y brindar seguridad.\\

En el ámbito de la seguridad, la video-vigilancia se define como el acto de observar una escena o escenas en busca de comportamientos específicos que podrian ser anormales o podrian indicar una posible emergencia o existencia de un comportamiento impropio \cite{NORMAN:201795}. Los sistemas de video-vigilancia en la actualidad, se han convertido en una herramienta esencial de la seguridad para mantener ``observado/vigilado'' un espacio muy importante para quien requiere el sistema; donde el mismo esta compuesto por un conjunto de cámaras, monitores y grabadoras los cuales forman parte esencial del sistema. Estos sistemas pueden ser instalados tanto en interiores como en exteriores de una propiedad o establecimiento, especialmente en lugares donde se desea mantener una vigilancia constante.\\

Gracias a la tecnología actual se ha podido automatizar la mayoría de las tareas que los humanos realizan y el campo de la video-vigilancia no ha sido la excepción. Con los continuos avances tecnológicos cada vez se desarrollan sistemas más completos y avanzados, permitiendo incrementar su eficacia y confiabilidad; por ejemplo la capacidad de poder vigilar en la oscuridad gracias a la tecnología de visión nocturna. Pero el campo más fascinante dentro de estos avances es el de la Inteligencia Artificial y específicamente la rama de la ``Visión por Computadora''. Gracias a las técnicas utilizadas en este campo de investigación, una computadora con el apoyo de algoritmos específicos y clasificadores, tiene la capacidad de identificar objetos, siluetas y/o elementos dentro de una escena captada por una cámara.\\

Estas nuevas características pueden ser explotadas en un sin fin de actividades diarias donde es necesaria la supervisión de una persona, permitiendo aún más la automatización de tareas de vigilancia. El problema a afrontar a partir de este escenario es evaluar si lo que esta siendo identificado en una escena representa un peligro para las personas.\\

\section{Antecedentes}
En la actualidad es común que empresas e instituciones tengan instalados sistemas de seguridad en sus ambientes como ser: oficinas, sitios de producción, almacenes, entradas, recepción, etc. pero realmente no solo las empresas tienen algún riesgo de situación de peligro o robo, si no también las personas en sus respectivos hogares.\\

Con el contínuo crecimiento del mercado de la seguridad, el precio de los equipos de video-vigilancia tendieron a decrecer pero aun no son accesibles para todo el mundo. Este hecho asociado con el incremento de la inseguridad independientemente de cada país, promueve los siguientes escenarios: un incremento en el uso de sistemas de video-vigilancia, sistemas con varias cámaras funcionando al mismo tiempo siendo monitoreadas solo por un usuario el cual no esta disponible todo el tiempo y la ausencia de características avanzadas de reconocimiento de escenas en sistemas de video-vigilancia convencionales.\

\section{Descripción del Problema}
Cuando el responsable de una casa esta ausente y nadie esta vigilando su hogar, la posibilidad de acontecer una situación anormal siempre esta presente. Si en el peor de los casos llegase a ocurrir algo en su hogar, esta persona solo se llega a enterarse si algún vecino se comunica con él para avisarle lo sucedido o en el peor de los casos, enterarse directamente a su regreso. Un sistema de video-vigilancia con las características de identificar movimiento y situaciones de peligro como ser: presencia de intrusos, fuego y humo, puede reducir el daño causado por los sucesos antes descritos por medio de la acción inmediata por parte del usuario en el momento de ser notificado, apoyado por la visualización en tiempo real de lo que estan captando las cámaras.\\

\subsection{Definición del problema}
Dificultad para advertir de forma inmediata situaciones de peligro en el hogar.\\

\section{Objetivos del Proyecto}
A continuación se presentan el objetivo general y los objetivos específicos.\\

\subsection{Objetivo General}
Facilitar la alerta inmediata ante situaciones de peligro en el hogar por medio de un sistema de video-vigilancia inteligente.\\

\subsection{Objetivos Específicos}
\begin{enumerate}
    \item Describir todos los factores que implican el proceso de transmisión de datos por la red.
    \item Especificar el proceso de análisis y procesamiento de imágenes con inteligencia artificial.
    \item Proveer una red neuronal para el reconocimiento y análisis de video.
    \item Identificar las partes que conforman el proceso de transmisión de video.
    \item Describir medios para la interacción entre la transmisión y el análisis de imágenes.
    \item Proveer el medio de acceso y notificación entre el sistema y el usuario.
\end{enumerate}

\section{Justificación}
El riesgo de que un suceso ponga en peligro la integridad física y material de las personas esta presente cada día y en cualquier lugar. A pesar de que esta posibilidad es imposible de eliminar, se pueden crear mecanismos que contrarresten el impacto que ocasionan dichos sucesos en los sitios que se quieren evitar. Algunas situaciones más comunes que pueden representar un peligro a la integridad física y/o material del hogar son: la presencia de intrusos en ausencia del responsable en el hogar y la presencia de fuego y/o humo en el interior y/o exterior del hogar.\\

Los sistemas de video-vigilancia permiten la visualización en tiempo real de lo que las cámaras estan captando, pero es necesario una persona que ejecute la acción constante de revisar dicha transmisión para identificar y alertar sobre algunas situaciones que según su criterio pueden llegar a ser peligrosas. Si la cantidad de cámaras es considerable, la eficacia del operador del sistema disminuye al tener que revisar la transmisión de varias cámaras.\\

Con el aprovechamiento de la tecnología actual se plantea la implementación de un prototipo para un sistema de video-vigilancia inteligente que permita retransmitir de manera remota los fotogramas captados por las cámaras, con la característica de alertar al usuario sobre los sucesos antes descritos una vez que se identifican por medio de técnicas de visión por computadora y redes neuronales, para la acción inmediata del usuario con el fin de disminuir su impacto.\\

\section{Alcances y límites}
\begin{itemize}
    \item La transmisión de fotogramas será implementado tanto para su ejecución en un ambiente local, como en línea.
    \item La notificación del evento identificado se realizará por medio de correo electrónico.
    \item La visualización en vivo del registro de la cámara se realizará por medio de un reproductor web de video con la capacidad de reproducir video en vivo HLS (HTTP Live Streaming).
    \item Se implementará la detección de: movimiento, silueta humana, fuego y humo.
    \item Los fotogramas capturados serán analizados por medio de técnicas de visión artificial y redes neuronales.
    \item La transmisión de video en tiempo real se realizará por medio de un servidor web que implementa software libre en el streaming de video.
\end{itemize}