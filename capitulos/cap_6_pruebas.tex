\chapter{Pruebas}

Según la metodologia ``cascada'', la fase final del desarrollo se cierra cuando se realizan pruebas sobre lo implementado en la fase anterior. Para probar en su totalidad el sistema de video-vigilancia se plantean las siguientes pruebas de aceptación.\\

\section{Prueba de conexión del módulo de cámaras}
En la siguiente tabla se describe la prueba realizada:\\

\begin{table}[H]
    \caption{Detalle de prueba de conexión de módulo de cámaras}
    \begin{center}
        \begin{tabular}{|>{\centering}p{0.3\textwidth}|m{0.6\textwidth}<{\centering}|} 
            \hline
            \textbf{Título de la prueba} & Única cámara conectada \\
            \hline
            \textbf{Descripción} & El servidor se encuentra en ejecución y una instancia del módulo de cámaras se conecta al servidor. No existen más cámaras conectadas.\\
            \hline
            \textbf{Comportamiento obtenido} & 
            \begin{itemize}
                \item El servidor registra la desconexión y lo muestra en consola.
                \item El servidor notifica al usuario por medio de correo electronico.
                \item La notificación provee información sobre la fecha y hora de la desconexión.
                \item La notificación detalla que hay más conexiones.
            \end{itemize} \\ 
            \hline
            \textbf{Estado de prueba} & Exitoso \\
            \hline
        \end{tabular}
    \end{center}
\end{table}

A continuación se muestra las capturas del comportamiento esperado:

En la siguiente tabla se describe otra de las prueba realizadas:\\

\begin{table}[H]
    \caption{Detalle de prueba de conexión del módulo de cámaras con más de una cámara disponible}
    \begin{center}
        \begin{tabular}{|>{\centering}p{0.3\textwidth}|m{0.6\textwidth}<{\centering}|} 
            \hline
            \textbf{Título de la prueba} & Más de una cámara conectada \\
            \hline
            \textbf{Descripción} & El servidor se encuentra en ejecución y una instancia del módulo de cámaras se conecta al servidor. Existen más de una cámara conectada.\\
            \hline
            \textbf{Comportamiento obtenido} & 
            \begin{itemize}
                \item El servidor acepta la conexión y lo muestra en consola.
                \item El servidor notifica al usuario por medio de correo electronico.
                \item La notificación provee información sobre la fecha y hora de la conexión, compartiendo el enlace para la transmisión en vivo.
                \item La notificación provee la misma información de las demás cámaras conectadas al servidor.
            \end{itemize} \\ 
            \hline
            \textbf{Estado de prueba} & Exitoso \\
            \hline
        \end{tabular}
    \end{center}
\end{table}

A continuación se muestra las capturas del comportamiento esperado:


\section{Prueba de desconexión del módulo de cámaras}

En la siguiente tabla se describe una de las pruebas realizadas:\\

\begin{table}[H]
    \caption{Detalle de prueba de desconexión del módulo de cámaras, única cámara conectada}
    \begin{center}
        \begin{tabular}{|>{\centering}p{0.3\textwidth}|m{0.6\textwidth}<{\centering}|} 
            \hline
            \textbf{Título de la prueba} & Única cámara conectada y se desconecta \\
            \hline
            \textbf{Descripción} & El servidor se encuentra en ejecución y la única instancia del módulo de cámaras se desconecta del servidor. No existen más cámaras conectadas.\\
            \hline
            \textbf{Comportamiento obtenido} & 
            \begin{itemize}
                \item El servidor registra la desconexión y lo muestra en consola.
                \item El servidor notifica al usuario por medio de correo electronico.
                \item La notificación provee información sobre la fecha y hora de la desconexión.
                \item La notificación detalla que no hay más conexiones.
            \end{itemize} \\ 
            \hline
            \textbf{Estado de prueba} & Exitoso \\
            \hline
        \end{tabular}
    \end{center}
\end{table}

En la siguiente tabla se describe otra de las pruebas realizadas:\\

\begin{table}[H]
    \caption{Detalle de prueba de desconexión de módulo de cámaras con más de una cámara conectada}
    \begin{center}
        \begin{tabular}{|>{\centering}p{0.3\textwidth}|m{0.6\textwidth}<{\centering}|} 
            \hline
            \textbf{Título de la prueba} & Una de las cámaras es desconectada \\
            \hline
            \textbf{Descripción} & El servidor se encuentra en ejecución y una instancia del módulo de cámaras se desconecta del servidor. Existen más cámaras conectadas.\\
            \hline
            \textbf{Comportamiento obtenido} & 
            \begin{itemize}
                \item El servidor acepta la conexión y lo muestra en consola.
                \item El servidor notifica al usuario por medio de correo electronico.
                \item La notificación provee información sobre la fecha y hora de la conexión, compartiendo el enlace para la transmisión en vivo.
            \end{itemize} \\ 
            \hline
            \textbf{Estado de prueba} & Exitoso \\
            \hline
        \end{tabular}
    \end{center}
\end{table}

A continuación se muestra las capturas del comportamiento esperado:


\section{Prueba de detección de fuego y notificación inmediata}

En la siguiente tabla se describe la prueba realizada sobre el detector:\\

\begin{table}[H]
    \caption{Detalle de prueba de detector de fuego}
    \begin{center}
        \begin{tabular}{|>{\centering}p{0.3\textwidth}|m{0.6\textwidth}<{\centering}|} 
            \hline
            \textbf{Título de la prueba} & Detector de fuego identifica incidencia \\
            \hline
            \textbf{Descripción} & El servidor se encuentra en ejecución y el proceso de identificación de fuego detecta una incidencia.\\
            \hline
            \textbf{Comportamiento obtenido} & 
            \begin{itemize}
                \item El servidor captura los fotogramas como prueba de la incidencia.
                \item El servidor notifica al usuario por medio de correo electronico.
                \item La notificación provee información sobre la fecha y hora de la incidencia, compartiendo el enlace para la transmisión en vivo.
            \end{itemize} \\ 
            \hline
            \textbf{Estado de prueba} & Exitoso \\
            \hline
        \end{tabular}
    \end{center}
\end{table}

A continuación se muestra las capturas del comportamiento esperado:

\section{Prueba de detección de silueta humana y notificación inmediata}

En la siguiente tabla se describe la prueba realizada sobre el detector:\\

\begin{table}[H]
    \caption{Detalle de prueba de detector de silueta humana}
    \begin{center}
        \begin{tabular}{|>{\centering}p{0.3\textwidth}|m{0.6\textwidth}<{\centering}|} 
            \hline
            \textbf{Título de la prueba} & Detector de silueta humana identifica incidencia \\
            \hline
            \textbf{Descripción} & El servidor se encuentra en ejecución y el proceso de identificación de silueta humana detecta una incidencia.\\
            \hline
            \textbf{Comportamiento obtenido} & 
            \begin{itemize}
                \item El servidor captura los fotogramas como prueba de la incidencia.
                \item El servidor notifica al usuario por medio de correo electronico.
                \item La notificación provee información sobre la fecha y hora de la incidencia, compartiendo el enlace para la transmisión en vivo.
            \end{itemize} \\ 
            \hline
            \textbf{Estado de prueba} & Exitoso \\
            \hline
        \end{tabular}
    \end{center}
\end{table}

A continuación se muestra las capturas del comportamiento esperado:


\section{Prueba de detección de movimiento y notificación inmediata}

En la siguiente tabla se describe la prueba realizada sobre el detector:\\

\begin{table}[H]
    \caption{Detalle de prueba de detector de movimiento}
    \begin{center}
        \begin{tabular}{|>{\centering}p{0.3\textwidth}|m{0.6\textwidth}<{\centering}|} 
            \hline
            \textbf{Título de la prueba} & Detector de movimiento identifica incidencia \\
            \hline
            \textbf{Descripción} & El servidor se encuentra en ejecución y el proceso de identificación de silueta humana detecta una incidencia.\\
            \hline
            \textbf{Comportamiento obtenido} & 
            \begin{itemize}
                \item El servidor captura los fotogramas como prueba de la incidencia.
                \item El servidor notifica al usuario por medio de correo electronico.
                \item La notificación provee información sobre la fecha y hora de la incidencia, compartiendo el enlace para la transmisión en vivo.
            \end{itemize} \\ 
            \hline
            \textbf{Estado de prueba} & Exitoso \\
            \hline
        \end{tabular}
    \end{center}
\end{table}

A continuación se muestra las capturas del comportamiento esperado:
