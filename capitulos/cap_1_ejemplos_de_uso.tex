\chapter{Introducción}

\begin{center}
    \textit{
        El presente proyecto demuestra la implementación de un sistema como propuesta de solución tecnológica para la alerta inmediata durante hechos/sucesos que pongan en peligro la integridad física y material en el hogar, como ser: fuego, humo, presencia de intrusos y violencia doméstica.
    }
\end{center}

La seguridad es un término usado para referirse a la ausencia de riesgo o a la confianza en algo o alguien; pero este panorama toma diversos sentidos según el campo en el que se referencia la seguridad. Aunque su objetivo consista
en reducir el riesgo a niveles aceptables, el mismo es inherente a cualquier actividad o situación y nunca podrá ser eliminado.

La seguridad en el hogar se convierte en un asunto sensible y delicado de tratar, únicamente por desarrollarse 
ser el espacio vital 
del ser humano. 

La seguridad es un tema muy de moda en la actualidad, con su presencia en muchos aspectos de
la vida cotidiana de las personas. La seguridad en el hogar, con el paso de los años se ha convertido en 
uno de los aspectos más importantes a tener en cuenta ante la posibilidad de que una situacion, que ponga en 
peligro nuestro hogar, pueda suceder.

Por ello, es preciso que antes de planificar la seguridad en la vivienda sepamos que queremos proteger
que es aquello que se halla más expuesto a riesgos y amenazas, tanto interiores como exteriores, 
y qué medidas y sistemas podemos implementar para garantizar su seguridad. \\

Enfocarnos en que debemos de proteger, focalizar nuestras principales prioridades y amenazas que acechan 
durante nuestra ausencia depende de nuestros propios hábitos cotidianos; pero hay situaciones que van mas
allá de lo previsto, situaciones que muchas veces no son tomadas en cuenta pero es probable que puedan suceder
 y mas aún cuando se esta ausente en el hogar donde no haya un responsable que pueda tratar el problema de manera inmediata.\\

Una de las amenazas más comunes en la actualidad es la delincuencia, y si uno no se encuentra en el hogar es posible que el robo se de con éxito. Otra de las situaciones más comunes son situaciones de violencia intrafamiliar o de algún desconocido hacia miembros de nuestra familia. También hay que tomar en cuenta los riesgos de incendio en cualquier hogar y que sin un accionar inmediato se pueden tener resultados lamentables para una familia.\\

\section{Antecedentes}
Los peligros están siempre presentes y es de vital importancia tratar de reaccionar de manera inmediata cuando una situación de peligro se presente en nuestro hogar. Pero resulta dificultoso saber si nuestro hogar o algún miembro de nuestra familia se encuentra en peligro en un determinado momento si uno esta asunte en ese momento.
Gracias a las redes neuronales y aprendizaje profundo; que son técnicas de inteligencia artificial es posible diseñar sistemas para el procesamiento de imágenes y aprendizaje para el reconocimiento de patrones de conducta violenta, humo, o intrusos en el hogar.
No solamente es necesario poder reconocer estas situaciones anteriormente mencionadas, sino también poder notificar en tiempo real de esta situación de manera inmediata después de ser identificada, cuando uno esta ausente en el hogar; gracias a la tecnología de transmisión de video en vivo es posible ver en tiempo relativamente real lo que esta sucediendo en ese momento, para poder realizar las acciones pertinentes

\section{Descripcion del Problema}
La ausencia en el hogar de los responsables familiares es común hoy en día por diferentes razones; 
estas pueden ser: trabajo, estudio, negocios, etc. 
Pero hay situaciones de peligro que ponen en riesgo la integridad física y material de la familia en las que es necesario una acción inmediata por parte del dueño de la casa, el padre de familia o el apoderado. Pero es difícil saber a exactamente si alguna situación de peligro extremo esta sucediendo en el hogar en ese instante.

\subsection{Definición del problema}
Dificultad para advertir de forma inmediata situaciones de peligro en el hogar.

\section{Objetivos del Proyecto}
A continuación se presentan el objetivo general y los objetivos específicos.

\subsection{Objetivo General}
Facilitar la alerta inmediata ante situaciones de peligro en el hogar por medio de un sistema de video-vigilancia inteligente.

\subsection{Objetivos Especificos}
\begin{enumerate}
    \item Describir todos los factores que implican el proceso de transmisión de datos por la red.
    \item Especificar el proceso de análisis y procesamiento de imágenes con inteligencia artificial.
    \item Proveer una red neuronal para el reconocimiento y análisis de video.
    \item Identificar las partes que conforman el proceso de transmisión de video.
    \item Describir medios para la interacción entre la transmisión y el análisis de imágenes.
    \item Proveer el medio de acceso y notificación entre el sistema y el usuario.
\end{enumerate}

\section{Justificación}
Cada día en nuestros hogares nos enfrentamos a situaciones de peligro que ponen en riesgo nuestra integridad física y material; poder reaccionar de manera inmediata ante estas situaciones es fundamental para que estos problemas afecten de manera de desmesurada a nuestro entorno.\\

La violencia es un problema que cada día va en aumento junto con la delincuencia. Si en nuestro hogar no se encuentra nadie o solo se encuentran nuestros familiares; es importante saber a cada momento si todo va bien en nuestra ausencia, pero no basta unas cuantas palabras para describirnos 
si todo va bien; ahí va el dicho que dice una imagen vale mas que mil palabras. En ese sentido se propone emplear la transmisión en vivo de imágenes para que el usuario pueda ver y verificar de que se trata de una situación que pone en riesgo la integridad de sus familiares.

\section{Alcances y límites}
\begin{itemize}
    \item El servicio de transmisión en vivo será implementado en un servidor en línea.
    \item Se desarrollara un aplicación móvil para la notificación de alguna situación de peligro.
    \item Se identificará: fuego, humo, personas no identificadas, y acciones violentas.
    \item Será posible poder ver en vivo lo que se esta trasmitiendo desde el hogar.
    \item Se procesará el video para reconocer situaciones de peligro.
    \item Se proveerá una red neuronal para procesar imágenes.
\end{itemize}

\subsection{Familia}
{\ttfamily typewriter (máquina de escribir)}

{\sffamily sans serif}

{\rmfamily roman}

\subsection{Forma}
\textbf{texto en negritas}\\
\textit{texto en itálicas}\\
\textsl{texto inclinado}\\
\texttt{texto en estilo máquina de escribir}\\
\textsc{texto en mayúsculas pequeñas}

\subsection{Tamaño}
{\tiny texto de prueba}\\
{\scriptsize texto de prueba}\\
{\footnotesize texto de prueba}\\
{\small texto de prueba}\\
{\normalsize texto de prueba}\\
{\large texto de prueba}\\
{\Large texto de prueba}\\
{\LARGE texto de prueba}\\
{\huge texto de prueba}\\
{\Huge texto de prueba}

\section{Listado}

\subsection{No numerados}
\begin{itemize}
    \item Item 1
    \item Item 2
    \item Item 3
\end{itemize}

\subsection{Numerados}
\begin{enumerate}
    \item Item 1
    \item Item 2
    \item Item 3
\end{enumerate}


\section{Referenciación con APA}

\subsection{Citación como parte de párrafo}

\subsection*{Un autor}

Como menciona \citeA{libro:ejemplo}, no es la única
forma de citar.\\

\subsection*{Varios autores}

Como menciona \citeA{libro:ejemplo_varios_autores}, no es la única
forma de citar.\\

\subsection{Citación en la parte final}

\subsection*{Un autor}

Duis fringilla tristique neque. Sed interdum libero ut metus.
Pellentesque placerat. Nam rutrum augue a leo. Morbi sed 
elit sit amet ante lobortis sollicitudin.Praesent blandit 
blandit mauris citumoris totalis. \cite{libro:ejemplo}.

\subsection*{Varios autores}

Duis fringilla tristique neque. Sed interdum libero ut metus.
Pellentesque placerat. Nam rutrum augue a leo. Morbi sed 
elit sit amet ante lobortis sollicitudin.Praesent blandit 
blandit mauris citumoris totalis. \cite{libro:ejemplo_varios_autores}.

\subsection{Citación con número de página}

Como menciona \citeA[p.~5]{libro:ejemplo}, no es la única
forma de citar.\\

Duis fringilla tristique neque. Sed interdum libero ut metus.
Pellentesque placerat. Nam rutrum augue a leo. Morbi sed 
elit sit amet ante lobortis sollicitudin.Praesent blandit 
blandit mauris citumoris totalis. \cite[p.~7-12]{libro:ejemplo_varios_autores}.

\subsection{Citación anexos}
Duis fringilla tristique neque. Sed interdum libero ut metus.
Pellentesque placerat (Ver Anexo A).

\section{Figuras}
Referenciando a la figura \ref{fig:ejemplo}.

\begin{figure}[H]
    \begin{center}
        \includegraphics[width=8cm]{img/capitulo_1/figura_ejemplo.png}
    \end{center}
    \caption{Explicación de la figura (Aquí)}
    Fuente: Adaptada de Apellido, N. (2000) \textit{Nombre del libro}.
    Editorial o universidad que lo publicó.
    \label{fig:ejemplo}
\end{figure}

\section{Tablas}

\subsection{Corto}
Referenciando a la tabla \ref{tabla:ejemplo}.\\

\begin{table}[H]
    \caption{Título de la tabla}
    \label{tabla:ejemplo}
    \begin{center}
        \begin{tabular}{c|c|c|c|}
            \cline{2-4}
            & \textbf{Columna 1} & \textbf{Columna 2} & \textbf{Columna 3} \\ \hline
            \multicolumn{1}{|c|}{\textbf{Fila 1}} & item               & item               & item               \\ \hline
            \multicolumn{1}{|c|}{\textbf{Fila 2}} & item               & item               & item               \\ \hline
            \multicolumn{1}{|c|}{\textbf{Fila 3}} & item               & item               & item               \\ \hline
        \end{tabular}
    \end{center}
    Nota. Extraída de Apellido, N. (2000) \textit{Nombre del libro}.
    Editorial o universidad que lo publicó.
\end{table}

\subsection{Multipágina}
Stique neque. Sed interdum libero ut metus.
Pellentesque placerat. Nam rutrum augue a leo. Morbi sed 
elit sit amet ante lobortis sollicitudin.Praesent blandit 
blandit mauris. Praesent lectus tellus, aliquet aliquam,
luctus a, egestas a, turpis.Mauris lacinia lorem sit amet
ipsum. Nunc quis urna dictum turpis accumsan semper. Tabla
\ref{tabla:tabla_largo_ejemplo}.\\

\begin{longtable}{p{2cm}|p{3cm}|p{3cm}|p{3cm}|p{3cm}|}
    \caption{Titulo de tabla multipágina} \label{tabla:tabla_largo_ejemplo} \\
    \cline{2-5}
    \multicolumn{1}{l|}{} & \multicolumn{1}{c|}{\textbf{Columna 1}} & \multicolumn{1}{c|}{\textbf{Columna 2}} & \multicolumn{1}{c|}{\textbf{Columna 3}} & \multicolumn{1}{c|}{\textbf{Columna 4}}\\ \hline
    \endfirsthead

    \multicolumn{5}{c}{{\tablename{} \thetable{} -- Continuación de tabla previa}} \\
    \multicolumn{5}{l}{} \\
    \cline{2-5}
    \multicolumn{1}{l|}{} & \multicolumn{1}{c|}{\textbf{Columna 1}} & \multicolumn{1}{c|}{\textbf{Columna 2}} & \multicolumn{1}{c|}{\textbf{Columna 3}} & \multicolumn{1}{c|}{\textbf{Columna 4}}\\ \hline
    \endhead

    \multicolumn{5}{c}{{Continua en la siguiente página.}} \\
    \endfoot

    \multicolumn{5}{l}{} \\
    \multicolumn{5}{l}{Nota. Extraída de Apellido, N. (2000) \textit{Nombre del libro}. Editorial o universidad que lo publicó.} \\
    \endlastfoot

    \multicolumn{1}{|c|}{\textbf{Fila 1}} & Lorem ipsum dolor sit amet, consectetuer adipiscing elit. & Lorem ipsum dolor sit amet, consectetuer adipiscing elit. & Lorem ipsum dolor sit amet, consectetuer adipiscing elit. & Lorem ipsum dolor sit amet, consectetuer adipiscing elit. \\ \hline
    \multicolumn{1}{|c|}{\textbf{Fila 2}} & Lorem ipsum dolor sit amet, consectetuer adipiscing elit. & Lorem ipsum dolor sit amet, consectetuer adipiscing elit. & Lorem ipsum dolor sit amet, consectetuer adipiscing elit. & Lorem ipsum dolor sit amet, consectetuer adipiscing elit. \\ \hline
    \multicolumn{1}{|c|}{\textbf{Fila 3}} & Lorem ipsum dolor sit amet, consectetuer adipiscing elit. & Lorem ipsum dolor sit amet, consectetuer adipiscing elit. & Lorem ipsum dolor sit amet, consectetuer adipiscing elit. & Lorem ipsum dolor sit amet, consectetuer adipiscing elit. \\ \hline
    \multicolumn{1}{|c|}{\textbf{Fila 4}} & Lorem ipsum dolor sit amet, consectetuer adipiscing elit. & Lorem ipsum dolor sit amet, consectetuer adipiscing elit. & Lorem ipsum dolor sit amet, consectetuer adipiscing elit. & Lorem ipsum dolor sit amet, consectetuer adipiscing elit. \\ \hline
    \multicolumn{1}{|c|}{\textbf{Fila 5}} & Lorem ipsum dolor sit amet, consectetuer adipiscing elit. & Lorem ipsum dolor sit amet, consectetuer adipiscing elit. & Lorem ipsum dolor sit amet, consectetuer adipiscing elit. & Lorem ipsum dolor sit amet, consectetuer adipiscing elit. \\ \hline
    \multicolumn{1}{|c|}{\textbf{Fila 6}} & Lorem ipsum dolor sit amet, consectetuer adipiscing elit. & Lorem ipsum dolor sit amet, consectetuer adipiscing elit. & Lorem ipsum dolor sit amet, consectetuer adipiscing elit. & Lorem ipsum dolor sit amet, consectetuer adipiscing elit. \\ \hline  
    \multicolumn{1}{|c|}{\textbf{Fila 7}} & Lorem ipsum dolor sit amet, consectetuer adipiscing elit. & Lorem ipsum dolor sit amet, consectetuer adipiscing elit. & Lorem ipsum dolor sit amet, consectetuer adipiscing elit. & Lorem ipsum dolor sit amet, consectetuer adipiscing elit. \\ \hline    
    \multicolumn{1}{|c|}{\textbf{Fila 8}} & Lorem ipsum dolor sit amet, consectetuer adipiscing elit. & Lorem ipsum dolor sit amet, consectetuer adipiscing elit. & Lorem ipsum dolor sit amet, consectetuer adipiscing elit. & Lorem ipsum dolor sit amet, consectetuer adipiscing elit. \\ \hline    
    \multicolumn{1}{|c|}{\textbf{Fila 9}} & Lorem ipsum dolor sit amet, consectetuer adipiscing elit. & Lorem ipsum dolor sit amet, consectetuer adipiscing elit. & Lorem ipsum dolor sit amet, consectetuer adipiscing elit. & Lorem ipsum dolor sit amet, consectetuer adipiscing elit. \\ \hline    
    \multicolumn{1}{|c|}{\textbf{Fila 10}} & Lorem ipsum dolor sit amet, consectetuer adipiscing elit. & Lorem ipsum dolor sit amet, consectetuer adipiscing elit. & Lorem ipsum dolor sit amet, consectetuer adipiscing elit. & Lorem ipsum dolor sit amet, consectetuer adipiscing elit. \\ \hline    
    \multicolumn{1}{|c|}{\textbf{Fila 11}} & Lorem ipsum dolor sit amet, consectetuer adipiscing elit. & Lorem ipsum dolor sit amet, consectetuer adipiscing elit. & Lorem ipsum dolor sit amet, consectetuer adipiscing elit. & Lorem ipsum dolor sit amet, consectetuer adipiscing elit. \\ \hline    
    \multicolumn{1}{|c|}{\textbf{Fila 12}} & Lorem ipsum dolor sit amet, consectetuer adipiscing elit. & Lorem ipsum dolor sit amet, consectetuer adipiscing elit. & Lorem ipsum dolor sit amet, consectetuer adipiscing elit. & Lorem ipsum dolor sit amet, consectetuer adipiscing elit. \\ \hline    
\end{longtable}

\section{Fórmulas matemáticas}

\section*{Simple}
\begin{equation}
    e^{i\pi} + 1 = 0
    \label{eq:euler}
\end{equation}

\section*{Matrices}
\begin{equation}
    \left[
    \begin{matrix}
     a & b & c \\
     d & e & f \\
     g & h & i
    \end{matrix}
    \right]
    \label{eq:matriz}
\end{equation}

\section*{Límites}
\begin{equation}
    \lim_{x\rightarrow\infty}\frac{3+x}{x^2} 
    \label{eq:limites}
\end{equation}

\section{Diagramas de flujo}

\begin{center}
    \begin{tikzpicture}[node distance = 3cm, auto]
        \footnotesize
        % Nodos
        \node [block] (init) {Inicio};
        \node [block, below of=init] (test) {Nodo A};
        \node [block, right of=test] (search) {Nodo B};
        \node [decision, below of=test] (test_done) {Decisión A};
        \node [block, below of=test_done] (intregate) {Nodo C};
        \node [cloud, left of=intregate] (code) {Extra};
        \node [decision, below of=intregate] (all_done) {Decisión B};
        \node [block, below of=all_done] (stop) {Nodo D};
        % Cordenadas
        \coordinate[left of=test] (il);
        \coordinate[left of=all_done] (al);
        % Líneas
        \path [line, dashed] (code) -- (intregate);
        \path [line, dashed] (test_done) -| node [near start] {si} (code);
        \path [line] (init) -- (test);
        \path [line] (test) -- (test_done);
        \path [line] (test_done) -- node {si}(intregate);
        \path [line] (test_done) -| node  [near start] {no} (search);
        \path [line] (intregate) -- (all_done);
        \path [line] (all_done.west) |- node [near end] {no} ([xshift=-2.3cm]al) -- ([xshift=-2.3cm]il)-- (test.west);
        \path [line] (all_done) -- node {si} (stop);
        \path [line] (search) -- (test);
    \end{tikzpicture}
\end{center}
