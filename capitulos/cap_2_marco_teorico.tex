\chapter{Marco Teórico}
En este capítulo, se presenta las principales tecnologías de detección en primer  plano del objeto en movimiento, descripción y extracción de características,  clasificación y reconocimiento del movimiento humano. Basado en el flujo óptico para la detección de objetos en movimiento. Como también la imagen de flujo de energía óptico para la detección de características de movimiento y se adoptaron redes neuronales convolucionales de región para elegir  características y reducir la dimensión. Luego, gracias al clasificador de máquina de vectores de soporte que puede ser entrenado y utilizado para clasificar y reconocer acciones; es posible distinguir efectivamente las acciones humanas y mejorar significativamente la precision del reconocimineto de las acciones humanas.

\section{Introducción}
Sistemas de videovigilancia inteligente La técnica clave del reconocimiento de la accion humana basada en la vision  por compoutadora consiste en describir y comprender los comportamientos humanos por medio de la vision por computadora.\\

Este proceso es una tarea complicada e integra algunos campos de investigacion que incluyen el procesamiento de imagen, aperndizaje automatico, reconocimiento de patrones, etc.\\

La detección de un objeto móvil consiste en separar las áreas de cambio en el video es decir en las imádgenes de fondo que comprenden el video, dicho de otra manera, separar correctamente las áreas y contornos del objetico movil. Es critico para el siguiente procesamiento la segementación efectiva \\

\section{Sistemas de videovigilancia}

\section{Visión por Computadora}

\section{Redes Neuronales}

\section{Protocolos de red IP/HTTP}

\section{Metodología de desarrollo}

\subsection{TCP/IP}

\subsection{HTTP}

\section{Video Streaming}

\subsection{Formatos}

\subsubsection{HLS}

\subsubsection{DASH}
