\chapter{Marco Teórico}

% En este capítulo, se presenta las principales tecnologías de detección en primer  plano del objeto en movimiento, descripción y extracción de características,  clasificación y reconocimiento del movimiento humano. Basado en el flujo óptico para la detección de objetos en movimiento. Como también la imagen de flujo de energía óptico para la detección de características de movimiento y se adoptaron redes neuronales convolucionales de región para elegir  características y reducir la dimensión. Luego, gracias al clasificador de máquina de vectores de soporte que puede ser entrenado y utilizado para clasificar y reconocer acciones; es posible distinguir efectivamente las acciones humanas y mejorar significativamente la precision del reconocimineto de las acciones humanas.

\section{Sistema de video vigilancia}
El término ``video-vigilancia'' es usado para hacer referencia al despliegue de cámaras de vídeo que cumplen el rol de videofilmadoras, las cuales guardan el contenido recolectado en un almacén digital, el cual puede ser visualizado en un monitor central. Entonces un sistema de video-vigilancia consiste en una instalación de seguridad cuya finalidad es el control y supervisión visual en tiempo real de instalaciones locales y remotas, mediante el uso de múltiples cámaras de vigilancia, así como de sistemas de visualización, grabación y archivo. Estos sistemas ayudan a proteger a las personas, bienes y recursos, mantienen una alerta activa y poseen un gran efecto disuasorio \cite{wikipedia:vvigilancia}.\\

El sistema llega a capturar imágenes y vídeos, que pueden ser comprimidos, almacenados, o enviados por una red de comunicación y pueden ser instalados en cualquier ambiente. En la figura \ref{fig:sistema_video_vigilancia} se visualiza el conjunto de elementos que forman un sistema de video vigilancia. Este sistema compone de un conjunto de cámaras que estan conectadas directamente a un grabador de video en red o N.V.R. (Network Video Recorder), el cual permite la visualización de las imágenes captadas por las cámaras en un monitor local y por medio de una conección a internet, permite la visualización en dispositivos externos a la red local.\\

\begin{figure}[H]
    \begin{center}
        \includegraphics[width=9cm]{img/capitulo_2/sis_videovigilancia.png}
    \end{center}
    \caption{Sistema actual de videovigilancia\\Fuente: Web}
    \label{fig:sistema_video_vigilancia}
\end{figure}

Existe una amplia oportunidad para el mercado de la video-vigilancia en todas las regiones del planeta especialmente en Asia y la región del Pacífico, debido a la apertura de pequeños negocios, como la construcción de áreas residenciales y ciudades ``inteligentes''. El mercado creciente de la vigilancia ha permitido que desarrolladores independientes y fabricantes diseñen nuevas implementaciones de sistemas de video-vigilancia, los cuales aplican nuevas características logradas por la tecnología actual.\\

En la figura \ref{fig:surveillance-market} se muestra como el mercado global de la video-vigilancia tuvo un valor de 42.9 billones de dólares en el 2019 y esta proyectado alcanzar a los 69.1 billones de dólares hasta el 2026; cuyo incremento registra una taza de crecimiento anual compuesto del 10\% desde el 2020 al 2026. \cite{marketsandmarkets:market-surveillance}\\

\begin{figure}[H]
    \begin{center}
        \includegraphics[width=13cm]{img/capitulo_2/surveillance-market.jpg}
    \end{center}
    \caption{Proyección del mercado de la video-vigilancia\\Fuente: MarketsAndMarkets(web)}
    \label{fig:surveillance-market}
\end{figure}

Lo más relevante del crecimiento de oportunidades, es la implementación de nuevas características en este tipo de sistemas; gracias a la implementación de técnicas de visión por computadora e inteligencia artificial (I.A.), como a la escalabilidad lograda por el uso de servicios basados en la nube. Las ramas de la inteligencia artificial como el Machine Learning(Aprendizaje Automático) y el Deep Learning(Aprendizaje profundo) permiten potenciar estas características.\\

Para el desarrollo del prototipo propuesto se implementan los siguientes componentes involucrados en el sistema:
\begin{itemize}
    \item Cámaras (denominados "nodos")
    \item Servidor TCP (Servicio que emplea el protocolo Tcp/Ip)
    \item Servidor HTTP (Servicio que emplea el protocolo Web)
    \item Módulo SMTP (Módulo de envio de correo electrónico)
\end{itemize}

Para la implementación del prototipo propuesto, es necesario describir el marco teórico relevante en la captura de fotogramas de video, transmisión de datos por el enlace tcp/ip, reconstrucción de información, consolidación y procesamiento de imágenes y transmisión de video en vivo. A continuación se detallan los conceptos teóricos relevantes anteriormente descritos.\\

\section{Arquitectura de red}
Una arquitectura de red es un esquema completo de comunicación entre computadoras, el cual provee: un esquema de trabajo, un diseño principal, la construcción y manejo de una red. <CITA>.La arquitectura de red más importante es la de Interconexión de Sistemas Abiertos (OSI), desarrollada por la Organización Internacional para la Estandarización (ISO).\\

La arquitectura OSI, es un estándar abierto para la comunicación en red a través de diferentes equipos y aplicaciones. Aunque no está ampliamente implementado, el modelo de 7 capas OSI es considerado el modelo de arquitectura de red principal para la intercomputación y comunicación entre redes. En la figura \ref{fig:osi} se puede apreciar el modelo OSI de 7 capas, detallando a continuación las siguientes capas:

\begin{enumerate}
    \item Capa física (Physical)
    \item Capa de enlace (Data Link)
    \item Capa de red (Network)
    \item Capa de transporte (Transport)
    \item Capa de sesión (Session)
    \item Capa de presentación (Presentation)
    \item Capa de aplicación (Aplication)
\end{enumerate}

Este modelo se organiza de la siguiente manera: las capas 7 a 4 se ocupan de las comunicaciones de extremo a extremo entre la fuente de datos y destinos, mientras que las capas 3 a 1 se ocupan de las comunicaciones entre los dispositivos de red. Por otro lado, las siete capas del modelo OSI pueden dividirse en dos grupos: \textbf{capas superiores} (capas 7, 6 y 5) y \textbf{capas inferiores} (capas 4, 3, 2, 1).\\

\begin{figure}[H]
    \begin{center}
        \includegraphics[width=10cm]{img/capitulo_2/capas.png}
    \end{center}
    \caption{Modelo OSI\\Fuente: <Libro>}
    \label{fig:osi}
\end{figure}

Su contraparte en las arquitecturas de red del modelo OSI, es TCP/IP, que no sigue exactamente el modelo OSI. Desafortunadamente, no existe un acuerdo universal sobre cómo describir TCP/IP con un modelo en capas. Generalmente se acepta que TCP/IP tiene menos niveles (de tres a cinco capas) que las siete capas del modelo OSI. En la figura \ref{fig:tcpip} se visualiza las capas que se adoptan en esta arquitectura. 

\begin{figure}[H]
    \begin{center}
        \includegraphics[width=8cm]{img/capitulo_2/tcp_ip_osi.jpg}
    \end{center}
    \caption{TCP/IP\\Fuente: Web}
    \label{fig:tcpip}
\end{figure}

La arquitectura TCP/IP omite algunas características que se encuentran en el modelo OSI, combina las características de algunas capas OSI adyacentes y separa otras capas. La estructura de 4 capas de TCP/IP(capa de aplicación, transporte, internet y acceso a la red)se construye a medida que la información se transmite de la capa de aplicacion a la capa de red física.\\

Cuando son envíados los datos, cada capa trata toda la información que recibe de la capa superior como datos, agrega información de control (encabezado) al frente de esos datos y luego los pasa a la capa inferior. Cuando se reciben los datos, se lleva a cabo el procedimiento opuesto ya que cada capa procesa y elimina su encabezado antes de pasar los datos a la capa superior.\\

\subsection{Protocolos}
El modelo OSI, y cualquier otro modelo de comunicación de red, proporciona solo un esquema conceptual para la comunicación entre computadoras, pero el modelo en sí mismo no proporciona métodos específicos de comunicación \textbf{<cita}. La comunicación real está definida por varios protocolos de comunicación.\\

En el contexto de la comunicación de datos, un protocolo es un conjunto formal de reglas, convenciones y estructuras de datos que determinan cómo las computadoras y otros dispositivos de red intercambian información a través de una red. Este método estándar permite la comunicación entre procesos (que potencialmente se ejecutan en diferentes equipos) y agrega un conjunto de reglas y procedimientos que deben respetarse para el envío y la recepción de datos a través de una red.\\

Similar a la manera de hablar el mismo lenguaje entre dos personas; un protocolo, simplifica la comunicación. La arquitectura de red proporciona solo un marco conceptual para la comunicación. El modelo no proporciona métodos específicos de comunicación, sino mas bien, la comunicación real está definida por varios protocolos de comunicación que son usados en la comunicación analógica y en la digital, y pueden ser usados en el procesos de transferencia de archivos y acceso a internet.\\

Los tipos mas comunes de protocolos de comunicación en red, incluyen:

\begin{itemize}
    \item \textbf{Automatización}: Estos protocolos se utilizan para automatizar diferentes procesos tanto en entornos comerciales como personales, como en edificios inteligentes, tecnología en la nube o vehículos autónomos.
    \item \textbf{Mensajeria Instantánea}: La comunicación basada en texto, en teléfonos inteligentes y computadoras suceden debido a una serie de diferentes protocolos de mensajeria instantánea.
    \item \textbf{Enrutamiento}: Protocolos de enrutamiento permiten la comunicación entre routers y otros dispositivos de red.
    \item \textbf{Transferencia de archivos}: Envio de archivos por medio de un canal de comunicacion.
    \item \textbf{Acceso a Internet}: El protocolo de Internet(IP) permite que los datos sean enviados entre dispositivos por medio de la red de internet.
\end{itemize}

A continuación se detalla algunos de los protocolos más conocidos:
\begin{itemize}
    \item \textbf{HTTP - Protocolo de transferencia de hipertexto}: Este protocolo de internet define la manera en la que los datos son enviados por internet y determina como los navegadores web y buscadores deben responder a determinados comandos.
    \item \textbf{SSH - Secure Socket Shell}: Este protocolo provee un acceso seguro al dispositivo, incluso si se encuentra en una red no segura. SSH es particularmente usado por administradores de red quienes manejan diferentes sistemas de manera remota.
    \item \textbf{SMS - Servicio de envio de mensajes cortos}: Este protocolo ha sido creado para enviar y recibir mensajes de texto sobre redes de telefonía celular. SMS refiere exclusivamente a mensajes basados en texto. Las imágenes, videos u otro contenido multimedia requiere el protocolo de Servicio de mensajeria multimedia (MMS), que es una extensión del protocolo SMS.
    \item \textbf{ICMP - Protocolo de control de mensajes de Internet}: Trabajo como un asistente del protocolo de Internet que se encarga de identificar fallos en la información y enviar mensajes de error hacia el usuario o servidor. Por ejemplo, si una dirección de este no está disponible o si una solicitud presenta fallas.
    \item \textbf{SMTP - Protocolo de transferencia de correo simple}: Este se encarga del intercambio de datos por texto en mensajes de correo electrónico entre ordenadores de una misma red.
\end{itemize}

\subsection{Modelo cliente-servidor}
El modelo cliente-servidor es una estructura de aplicación distribuida que divide tareas o cargas de trabajo entre los proveedores de un recurso o servicio, denominados servidores, y los solicitantes del servicio, denominados clientes.<cita 1> \\

A menudo, los clientes y los servidores se comunican a través de una red informática en hardware independiente, pero tanto el cliente como el servidor pueden residir en el mismo sistema.\\

Un host de servidor ejecuta uno o más programas de servidor, que comparten sus recursos con los clientes. Un cliente normalmente no comparte ninguno de sus recursos, pero solicita contenido o servicio de un servidor. En la figura \ref{fig:client_server} se aprecia un diagrama que representa el modelo cliente-servidor, donde los clientes acceden al servicio del servidor por medio de la red internet.\\

\begin{figure}[H]
    \begin{center}
        \includegraphics[width=10cm]{img/capitulo_2/client-server.jpeg}
    \end{center}
    \caption{Modelo cliente-servidor\\Fuente: Web}
    \label{fig:client_server}
\end{figure}

Por lo tanto, los clientes, inician sesiones de comunicación con los servidores, que esperan las solicitudes entrantes. Para mencionar algunos ejemplos de aplicaciones informáticas que utilizan este modelo cliente-servidor son el correo electrónico, la impresión en red y la World Wide Web (Internet).\\

La comunicación entre el cliente y el servidor se realiza por medio de una conexión o enchufe (socket) el cual define la dirección y puerto por la cual van a ser enviados y/o recibidos los datagramas entre ambos actores.


Un servidor puede recibir solicitudes de muchos clientes distintos en un período corto. Una computadora solo puede realizar una cantidad limitada de tareas en cualquier momento y depende de un sistema de programación para priorizar las solicitudes entrantes de los clientes para acomodarlas. Para evitar abusos y maximizar la disponibilidad, el software del servidor puede limitar la disponibilidad para los clientes. Los ataques de denegación de servicio están diseñados para explotar la obligación de un servidor de procesar solicitudes al sobrecargarlo con tasas de solicitud excesivas. Se debe aplicar el cifrado si se va a comunicar información confidencial entre el cliente y el servidor.

% El protocolo de Control de Transmisión (TCP) es uno de los protocolos fundamentales de internet. Fue creado entre los años 1973-1974 por Vint Cerf y Robert Kahn. Este es uno de los principales protocolos de la ``capa de transporte'' del modelo TCP/IP.<cita> A nivel de aplicación, posibilita la administración de datos que vienen del nivel más bajo del modelo, o van hacia él, (es decir, al protocolo IP). Cuando se proporcionan los datos al protocolo IP, este los agrupa en datagramas IP, fijando el campo del protocolo en 6 (para que sepa con anticipación que el protocolo es TCP). \\

% El protocolo TCP verifica los datos que son enviados por Internet, mientras que el segundo, IP o Internet protocol, se encarga de enviar esos datos a su destino suministrándoles un encabezado para identificarlos. Tanto uno como el otro son protocolos diferentes, que incluso habitan en capas diferentes del modelo OSI, pero son tan importantes el uno para el otro, que suele llamárseles TCP/IP como si fueran uno solo porque trabajan en conjunto y se necesitan. TCP e IP son la base de todo el Internet que conocemos actualmente.\\

% Las principales características del protocolo TCP son las siguientes: 

% \begin{itemize}
%     \item Ofrece soporte a la mayoria de las aplicaciones más populares de internet, incluidas HTTP, SMTP, SSH y FTP. 
%     \item Organiza los datagramas nuevamente en orden cuando vienen del protocolo IP.
% \end{itemize}



\begin{figure}[H]
    \begin{center}
        \includegraphics[width=7cm]{img/capitulo_2/tcp.png}
    \end{center}
    \caption{Flujo de interacción entre servidor y cliente TCP\\Fuente: Web}
    \label{fig:tcp_flow}
\end{figure}

En el lado del cliente: el cliente conoce el nombre de host de la máquina en la que se ejecuta el servidor y el número de puerto en el que escucha el servidor. Para realizar una solicitud de conexión, el cliente intenta reunirse con el servidor en la máquina y el puerto del servidor. El cliente también necesita identificarse ante el servidor para vincularse a un número de puerto local que utilizará durante esta conexión. Normalmente lo asigna el sistema.

\begin{figure}[H]
    \begin{center}
        \includegraphics[width=7cm]{img/capitulo_2/socket_request.jpg}
    \end{center}
    \caption{Solicitud de conexión por medio de un socket\\Fuente: Oracle}
    \label{fig:tcp_flow}
\end{figure}

Si todo va bien, el servidor acepta la conexión. Tras la aceptación, el servidor obtiene un nuevo socket vinculado al mismo puerto local y también tiene su punto final remoto establecido en la dirección y el puerto del cliente. Necesita un nuevo socket para que pueda continuar escuchando el socket original para las solicitudes de conexión mientras atiende las necesidades del cliente conectado.

\begin{figure}[H]
    \begin{center}
        \includegraphics[width=7cm]{img/capitulo_2/socket_connection.jpg}
    \end{center}
    \caption{Conexión establecida entre sockets\\Fuente: Oracle}
    \label{fig:tcp_flow}
\end{figure}

En el lado del cliente, si se acepta la conexión, se crea correctamente un socket y el cliente puede usar el socket para comunicarse con el servidor. El cliente y el servidor ahora pueden comunicarse escribiendo o leyendo desde sus sockets.

Un punto final es una combinación de una dirección IP y un número de puerto. Cada conexión TCP se puede identificar de forma única por sus dos puntos finales. De esa manera, puede tener múltiples conexiones entre su host y el servidor.
Los clientes y servidores intercambian mensajes en un patrón de mensajería de solicitud-respuesta. El cliente envía una solicitud y el servidor devuelve una respuesta. Este intercambio de mensajes es un ejemplo de comunicación entre procesos. Para comunicarse, las computadoras deben tener un lenguaje común y deben seguir reglas para que tanto el cliente como el servidor sepan qué esperar. El idioma y las reglas de comunicación se definen en un protocolo de comunicaciones. Todos los protocolos operan en la capa de aplicación. El protocolo de la capa de aplicación define los patrones básicos del diálogo. Para formalizar aún más el intercambio de datos, el servidor puede implementar una interfaz de programación de aplicaciones (API).[3] La API es una capa de abstracción para acceder a un servicio. Al restringir la comunicación a un formato de contenido específico, facilita el análisis. Al abstraer el acceso, facilita el intercambio de datos entre plataformas.[4]
\subsection{HTTP}
El protocolo de transferencia de hipertexto (HTTP) es el protocolo más utilizado en Internet. Es el protocolo usado en cada transacción de la Web (www). Este protocolo permite la transferencia de archivos (principalmente, en formato HTML) entre un navegador (el cliente) y un servidor web.\\

HTTP define la sintaxis y la semántica que utilizan los elementos software de la arquitectura web (clientes, servidores, proxies) para comunicarse. Es un protocolo orientado a transacciones y sigue el esquema ``petición - respuesta'' entre un cliente y un servidor. 

\begin{figure}[H]
    \begin{center}
        \includegraphics[width=9cm]{img/capitulo_2/http1.jpg}
    \end{center}
    \caption{Protocolo HTTP\\Fuente: Web}
    \label{fig:sistema_video_vigilancia}
\end{figure}


El protocolo tiene las siguientes partes:
\begin{itemize}
    \item Al cliente que efectúa la petición (un navegador o un spider) se lo conoce como "user agent" (agente del usuario). 
    \item A la información transmitida se la llama recurso y se la identifica mediante una cadena de caracteres denominada dirección URL.
    \item Los recursos pueden ser archivos, el resultado de la ejecución de un programa, una consulta a una base de datos, la traducción automática de un documento, etc.
\end{itemize}

<en la imagen???>

\section{Inteligencia Artificial}
La Inteligencia Artificial (I.A.), como área de las ciencias de la computación, en los últimos tiempos dejó de estar reservada para la investigación y ha formado parte del desarrollo de la sociedad.\\

El cerebro es el órgano más increible del cuerpo humano; establece la forma en la que percibimos las imágenes, sonido, olores, sabores y el tacto; por lo tanto permite al ser humano almacenar recuerdos, experimentar emociones e incluso soñar. Sin él, el ser humano sería un organismo primitivo, incapaz de otra cosa que el más simple de los reflejos. Por lo tanto el cerebro es lo que hace a este ser, un ser inteligente.<cita??????>\\

Durante décadas se ha investigado para construir máquinas inteligentes con cerebros como el del ser humano; asistentes robotizados para limpiar los hogares, coches que se conducen por solos, microscopios que detecten enfermedades automáticamente. Pero en la construcción de estas máquinas artificialmente inteligentes se presentan problemas computacionales complejos; problemas que el cerebro humano puede resolver en una fracción de segundos. Las formas de analizar y resolver este tipo de problemas, son estudiados por la inteligencia artificial.

\begin{figure}[H]
    \begin{center}
        \includegraphics[width=8cm]{img/capitulo_2/ia.png}
    \end{center}
    \caption{Campo de acción de la Inteligencia Artificial.\\Fuente: Web}
    \label{fig:ia}
\end{figure}

A menudo los términos Inteligencia Artificial, Aprendizaje Automático (Machine Learning) y Aprendiazaje Profundo (Deep Learning) son usados de manera indistinta, pero se debe tener en cuenta su significado diferente. Por los años `80 la Inteligencia Artificial era una característica que se alcanzaba al definir un conjunto de reglas que decían que hacer en un determinado momento, de esta manera un sistema `inteligente' solo obedecía reglas de acción programadas \cite{researchgate:ia}. En la figura \ref{fig:ia} se ilustra como la Inteligencia Artificial engloba a sus subcampos de estudio como ser el Machine Learning y el Deep Learning.

\subsection{Redes Neuronales}
Las redes neuronales artificiales son un modelo inspirado en el funcionamiento del cerebro humano. Está formado por un conjunto de nodos conocidos como neuronas artificiales que están conectadas y transmiten señales entre sí. Estas señales se transmiten desde la entrada hasta generar una salida. En la figura \ref{fig:classical_ml} se aprecia la estructura propia de una neurona artificial.\\

\begin{figure}[H]
    \begin{center}
        \includegraphics[width=7cm]{img/capitulo_2/neurona.png}
    \end{center}
    \caption{Ilustración de una Neurona artificial
        \\Fuente: Web}
    \label{fig:classical_ml}
\end{figure}

El objetivo principal de un modelo neuronal, es aprender modificándose automáticamente a si mismo, llegando a realizar tareas complejas que no podrían ser realizadas mediante la clásica programación basada en reglas. De esta forma se pueden automatizar funciones que al principio solo podrían ser realizadas por personas. Con su semejanza al del cerebro humano, las redes reciben una serie de valores de entrada y cada una de estas entradas llega a un nodo llamado neurona.\\

Las neuronas de la red están a su vez agrupadas en capas que forman la red neuronal. Cada una de las neuronas de la red posee a su vez un peso, un valor numérico, con el que modifica la entrada recibida. Los nuevos valores obtenidos salen de las neuronas y continúan su camino por la red. Este funcionamiento puede observarse de forma esquemática en la figura \ref{fig:estructura_red_neuronal}.\\

\begin{figure}[H]
    \begin{center}
        \includegraphics[width=7cm]{img/capitulo_2/Redes_neuronales_esquema.png}
    \end{center}
    \caption{Ilustración de una Red Neuronal
        \\Fuente: Web}
    \label{fig:estructura_red_neuronal}
\end{figure}

Una vez que se alcanza el final de la red se obtiene una salida que será la predicción calculada por la red. Cuantas más capas posea la red y más compleja sea, también serán mas complejas las funciones que pueda realizar. Para conseguir que una red neuronal realice las funciones deseadas, es necesario \textbf{entrenarla}.\\

El entrenamiento de una red neuronal se realiza modificando los pesos de sus neuronas para que consiga extraer los resultados deseados. Para ello lo que se hace es introducir datos de entrenamiento en la red, en función del resultado que se obtenga, se modifican los pesos de las neuronas según el error obtenido y en función de cuanto haya contribuido cada neurona a dicho resultado. Este método es conocido como Backpropagation o propagación hacia atrás. Con este método se consigue que la red aprenda, consiguiendo un modelo capaz de obtener resultados muy acertados incluso con datos muy diferentes a los que han sido utilizados durante su entrenamiento \cite{atriainnovation:ia}.\\

Estas redes neuronales son utilizadas para tareas de predicción y clasificación. Esta técnica se ha convertido en una pieza clave para el desarrollo de la Inteligencia Artificial, como se describió previamente es uno de los principales campos de investigación y el que más esta evolucionando con el tiempo, ofreciendo cada vez soluciones más complejas y eficientes.

\subsubsection{Redes neuronales convolucionales}
Dentro del conjunto de tipos de redes neuronales tenemos las convolucionales, que específicamente serviran en el campo de la visión artificial. Las redes neuronales convolucionales son un algoritmo de Aprendizaje Profundo (Deep Learning) que está diseñado para trabajar con imágenes, tomando estas como entrada, asignándoles importancias (pesos) a ciertos elementos en la imagen para así poder diferenciar unos de otros.\\

\begin{figure}[H]
    \begin{center}
        \includegraphics[width=10cm]{img/capitulo_2/convolucional.png}
    \end{center}
    \caption{Ilustración de una Red Neuronal Convolucional
        \\Fuente: Web}
    \label{fig:red_neuronal_convolucional}
\end{figure}

Este es uno de los principales algoritmos que ha contribuido en el desarrollo y perfeccionamiento del campo de visión por computadora. Las redes convolucionales contienen varias capas ocultas como se ilustra en la figura \ref{fig:estructura_red_neuronal}, donde las primeras puedan detectar líneas, curvas y así se van especializando hasta poder reconocer formas complejas como un rostro, siluetas, etc. \cite{convolutional:ia}. \\

\begin{figure}[H]
    \begin{center}
        \includegraphics[width=3cm]{img/capitulo_2/kernel.png}
    \end{center}
    \caption{Movimiento del Kernel
        \\Fuente: Web}
    \label{fig:kernel}
\end{figure}

El proceso que se distingue de estas redes son las convoluciones. El cual consiste en tomar un grupo de píxeles de la imagen de entrada e ir realizando un producto escalar con un kernel. El kernel recorrerá todas las neuronas de entrada y obtendremos una nueva matriz, la cual será una de las capas ocultas. En el caso de que la imagen sea de color se tendrán 3 kernels del mismo tamaño que se sumarán para obtener una imagen de salida.\\

El kernel en las redes convolucionales se considera como un filtro que se aplica para extraer ciertas características importantes o patrones de esta y es usado para detectar bordes, enfoque, desenfoques, entre otras características de la imagen y es logrado para la convolución entre la imagen y el kernel.

Este proceso se desarrolla entre la imagen y el kernel, con la finalidad de que este filtro o kernel recorra toda la imagen (pixel por pixel). Por lo general, el kernel es de menor tamaño que la imagen. La convolución permite multiplicar el kernel con la porción de imagen escogida, realiza un producto escalar a medida que el kernel se va desplazando; por esta razón es un proceso iterativo .\\

Es una operación que se usa en las redes convolucionales. El padding se aplica agregando píxeles de valor cero alrededor de la imagen original.
Tiene dos usos:
El primero es para que al realizar la convolución la imagen resultante sea de igual tamaño que la imagen original.
El segundo es cuando se tiene información relevante en las esquinas de la imagen por lo que al realizar convolución el filtro pasa más por el centro de la imagen que en las esquinas, por lo que se aplica el padding para tener la información más relevante cerca del centro.\\

Las tareas comunes de este tipo de redes son: detección o categorización de objetos, clasificación de escenas y clasificación de imágenes en general. La red toma como entrada los pixeles de una imagen. 

\section{Machine Learning (Aprendizaje de Máquina)}
Es un subcampo de la Inteligencia Artificial cuyo objetivo es entender la estructura de la información y ajustar estos datos en modelos que puedan ser entendidos y utilizados por las personas. \cite{digitalocean:machinelearning}.\\

A diferencia de la computación tradicional, donde los algoritmos resuelven problemas específicos, los algoritmos de Machine Learning entrenan a las computadoras con datos de entrada y emplean análisis estadístico para generar valores de salida que se clasifican según a un rango específico. Por eso el Machine Learning facilita a las computadoras construir modelos a partir de datos ejemplo para automatizar el proceso de toma de decisiones basados en estos datos.\\

\begin{figure}[H]
    \begin{center}
        \includegraphics[width=10cm]{img/capitulo_2/machinelearning.png}
    \end{center}
    \caption{Diferencias entre programación clásica y M.L.\\Fuente: Deep Learning with Python}
    \label{fig:classical_ml}
\end{figure}

En la figura \ref{fig:classical_ml} se aprecia la diferencia y similitud entre la programación clásica de la inteligencia artificial de los años `80 y las novedosas técnicas del aprendizaje automático. La programación clásica necesita de reglas y datos de entrada para que esta funcione como un sistema inteligente y pueda dar una respuesta, mientras que el Machine Learning necesita de datos y sus respectivas respuestas esperadas, para identificar patrones que los relacionen; y de esta manera  desarrollar reglas que generan respuestas para nuevos datos.

\subsection{Métodos de Machine Learning}
En el Machine Learning, las tareas son generalmente clasificadas en grandes categorias, las cuales estan basadas en el modo en el que el ``aprendizaje'' es ejecutado.\\

Los métodos más adoptados en el Machine Learning son: el aprendizaje supervisado, que entrena un algoritmo basado en un ejemplo de entrada y salida el cual esta categorizado por un humano, y el aprendizaje no supervisado, que proporciona el algoritmo sin ningún dato categorizado permitiendo encontrar una estructura dentro de los datos de entrada.\

\subsubsection{Aprendizaje Supervisado}
La computadora esta provista con entradas de ejemplo las cuales se categorizan con sus respectivas salidas esperadas. El propósito de este metodo consiste en que el algoritmo pueda  ``aprender'' comparando la actual salida con las salidas esperadas para encontrar errores y en consecuencia modificar el modelo. El aprendizaje supervisado por lo tanto usa patrones para predecir valores categorizados en datos no categorizados.\\
<Posible imagen>

En la ilsutracion.....

\subsubsection{Aprendizaje No Supervisado}
La información provista a la computadora no está categorizada, por lo que los algoritmos de aprendizaje buscan similitudes entre los datos de entrada. Como los datos no etiquetados son más abundantes que los datos etiquetados, los métodos de aprendizaje automático que facilitan el ``aprendizaje'' pasan a ser más importantes.\

<Mas conceptos> <ref bibliografica>

\section{Deep Learning (Aprendizaje profundo)}
Según la figura \ref{fig:ia}, el aprendizaje profundo es un subcampo dentro del Machine Learning, el cual hace uso de distintas redes neuronales para lograr el ``aprendizaje'' de sucesivas capas de representación que son relevantes para los datos.\\

El término Deep ``profundo'', hace referencia a la cantidad de capas de representación que se utilizan en un modelo; en general es posible utilizar decenas incluso cientos capas de representación, los cuales aprenden de forma automática a medida que el modelo es entrenado con los datos \cite{iaarbook:artificialvision}.

\section{Técnicas de visión por computadora}
La visión por computadora es una técnica de recolección de información que surge por la inspiración en el sistema visual humano, el cual es la principal fuente de información para el cerebro. Su meta es de modelar y automatizar el proceso de reconocimiento visual de objetos en la vida real.\\

De los cinco sentidos que poseen las personas, la vista es la más importante. Por lo tanto la visión, es una tarea de procesamiento de información; pero tiene un grado de complejidad elevado, ya que para saber que es lo qué hay en el mundo nuestros cerebros deben ser capaces de representar esta información en toda su abundancia de color, forma, movimiento, detalle y belleza. \cite{iaarbook:artificialvision}\\

Por lo tanto, la visión por computadora o visión artificial compone de un conjunto de herramientas y métodos que permiten obtener, procesar y analizar imágenes del mundo real, con el objetivo de ser tratadas por una computadora. Estos métodos van a permitir automatizar un amplio conjunto de tareas al aportar a las computadoras información que es necesaria para la toma de desiciones en sus tareas asignadas. La visión por computadora trata de imitar a la visión humana, usando geometría y un enfoque estadístico para tratar el problema.\\

\subsection{Aplicaciones}
Esta rama de la Inteligencia Artificial aún sigue en investigación y mejoras donde sus aplicaciones más comunes son:

\begin{itemize}
    \item \textbf{Reconocimiento óptico de caracteres:} Detección automática de símbolos que pertenecen a un alfabeto.
    \item \textbf{Inspección robotizada:} Revisión rápida de piezas para garantizar la calidad de componentes fabricados.
    \item \textbf{Modelado 3D:} Construcción de modelos 3D a partir de fotografías.
    \item \textbf{Imágenes médicas:} Análisis de radiografías.
    \item \textbf{Conducción segura:} Detección de obstáculos por medio de un sistema de conducción asistida por cámaras.
    \item \textbf{Vigilancia:} Monitoreo de intrusos, análisis del tráfico vial, monitoreo de piscinas, etc.
    \item \textbf{Detección de rostros:} Mediante algoritmos de reconocimiento facial se reconocen rostros usados en métodos de biometría.
\end{itemize}

\subsection{Librerias}
Una de las librerias mas utilizadas para las técnicas de vision por computadora es OpenCV. Es una biblioteca de uso libre para el desarrollo de aplicaciones usando visión artificial desarrollada por Intel. Esta libreria reune diversas caracteristicas que la hacen popular, por ejemplo: 
\begin{itemize}
    \item Permite su uso para fines comerciales y de investigación.
    \item Se encuentra disponible par varias plataformas como ser GNU/Linux, Mac OS, Windows y Android.
    \item Documentación completa y explicada, con una comunidad de desarrolladores activa.
    \item El procesamiento de imágenes en su escalado, eliminación de ruido y formateo de imagen y video.
    \item El uso y modificación de sus 2500 modelos pre-optimizados que son incluidos en la libreria, acorde a las necesidades del usuario.
    \item El uso del estado del arte de modelos de visión por computadora como también de aprendizaje de máquina (Machine Learning).
    \item El desarrollo de modelos en varias categorías de investigación como ser: reconocimiento facial, detección y seguimiento de objetos, extracción de modelos 3D, etc.
\end{itemize}

\begin{figure}[H]
    \begin{center}
        \includegraphics[width=3cm]{img/capitulo_2/cv2_logo.png}
    \end{center}
    \caption{Logotipo de la libreria\\Fuente: Web}
    \label{fig:cv2_logo}
\end{figure}

Una de las características mas interesantes de OpenCV es el reconocimiento facial. OpenCV, en su extensa biblioteca de funciones, brinda las capacidades para realizar las tareas de preprocesamiento sin ningún problema, así como los algoritmos de predicción. Además de usar el algoritmo de detección de objetos, es posible usar el seguimiento de objetos, para identificar rostros en una transmisión de video. OpenCV incluso posee funciones para configurar fácilmente el modelo en una transmisión en vivo, como en un video pregrabado \cite{medium:opencv}. 

Existen otras librerias que no son tan populares y representan un pago adicional.

% \section{Redes Neuronales}
\section{Transmision de video en vivo}
<figuras> y conceptos

\subsection{Protocolos}
<figuras> y conceptos

\subsubsection{HLS}
\begin{figure}[H]
    \begin{center}
        \includegraphics[width=12cm]{img/capitulo_2/hls.jpg}
    \end{center}
    \caption{HLS\\Fuente: web}
    \label{fig:classical_ml}
\end{figure}

\subsubsection{DASH}
\begin{figure}[H]
    \begin{center}
        \includegraphics[width=12cm]{img/capitulo_2/dash.png}
    \end{center}
    \caption{Dash\\Fuente: web}
    \label{fig:classical_ml}
\end{figure}


\section{Lenguaje de Programacion}
Python es un lenguaje de programación interpretado cuya filosofía hace hincapié en la legibilidad de su código. Se trata de un lenguaje multiparadigma, ya que soporta parcialmente la orientación a objetos, programación imperativa y, en menor medida, programación funcional. Es un lenguaje interpretado, dinámico y multiplataforma.\\

<logo>

Python usa tipado dinámico y conteo de referencias para la administración de memoria. Una característica importante de Python es la resolución dinámica de nombres; es decir, lo que enlaza un método y un nombre de variable durante la ejecución del programa (también llamado enlace dinámico de métodos).\\

<imagen de popularidad>
Motivo
Es usado en Machine Learning y Deep Learning es el lenguaje lider en la inteligencia artificial

\section{Metodología de desarrollo}
El término de ingenieria de software se toma propone por primera vez en el conjunto de conferencias históricas organizadas por el comité de ciencia de la OTAN.\footnote{Organización del Tratado del Atlántico Norte.} en los años 60. Para ese tiempo la ingenieria de software tampoco era conocida ni aceptada donde en ese entonces los proyectos de software complejos eran problemáticos y costosos de completar donde se supuso que sería beneficioso considerar el desarrollo de software como ingeniería \cite{Ganis}.\\

Los encargados de la codificación del software denominados programadores, en un principio eran ingenieros y como el costo computacional era alto, se utilizó un concepto de hardware denominado "mide dos veces, corta una vez"\cite{Ganis}.\\
La naturaleza cautelosa de esta costumbre provocó el desarrollo de metodologías que permitieron a los equipos de proyectos creen planes lentos y metódicos para la creación de sistemas de software.\\

\subsection{Modelo Cascada (Waterfall)}
En el inicio de su definición como un modelo para el desarrollo de software; este concepto fue abordado por el Dr. Winston Royce, por medio de un artículo escrito sobre la gestión y dirección de proyectos grandes y complejos de software \cite{Winston}. En ese artículo basandose en experiencias de desarrollo de software para la planificacion de misiones aereas; el autor describe los fundamentos del desarrollo de software. Gran parte de esos fundamentos aun son aplicables en la actualidad. En el planteamiento de estos fundamentos, Royce presenta un conjunto de fases los cuales forman parte de una secuencia de desarrollo de software ilustrada en la figura \ref{fig:cascada}.

\begin{figure}[H]
    \begin{center}
        \includegraphics[width=12cm]{img/capitulo_2/cascada2.png}
    \end{center}
    \caption{Modelo Cascada}
    \centering{Fuente: Elaboracion Propia}
    \label{fig:cascada}
\end{figure}

Una vez definida esta secuencia, se creó el concepto de ``Cascada", como un modelo de desarrollo con actividades bien definidas y organizadas con un objetivo independiente dando origen al diseño del primer S.D.M.\footnote{Metodología de Desarrollo de Software} \cite{Bell&Thayer}. En la figura \ref{fig:cascada} la fase de análisis y la de codificación entregan la mayor parte del producto esperado, mientras que las otras fases estan puestas para ser organizadas y planificadas de manera independiente para un mejor manejo de los recursos del proyecto.\\

De acuerdo al modelo Cascada, se enfatiza en la dependecia secuencial del producto entregado en el paso previo. Es decir existe una dependencia que mantiene en espera el diseño del sistema mientras que el análisis del modelo no sea aprobado o concluido y consecuentemente la fase de codificación se verá en espera también hasta que el diseño se concluya.\\

Cada una de las fases guarda una relacion iterativa con el siguiente paso definido en la metodología que asegura la completitud del producto entregado en la fase antetior. Esta relación esta ilustrada en la figura \ref{fig:cascada_iterativa}. Al final de cada etapa, el modelo está diseñado para llevar a cabo una revisión final, que se encarga de determinar si el proyecto está listo para avanzar a la siguiente fase. Este modelo fue el primero en originarse y es la base de todos los demás modelos de ciclo de vida dentro de un proyecto de desarrollo de software.\\

\begin{figure}[H]
    \begin{center}
        \includegraphics[width=12cm]{img/capitulo_2/cascada_iterativa.png}
    \end{center}
    \caption{Modelo Cascada: Relación iterativa entre las fases sucesivas.}
    \centering{Fuente: Elaboracion Propia}
    \label{fig:cascada_iterativa}
\end{figure}

Existen diferentes versiones de las fases del modelo en cascada y según la versión o enfoque, la cantidad de fases puede variar. Sin embargo las principales son las siguientes:

\begin{enumerate}
    \item Requerimientos del sistema
    \item Requerimientos de Software
    \item Análisis
    \item Diseño del programa
    \item Codificación
    \item Pruebas
    \item Mantenimiento 
\end{enumerate}

A continuacion se detalla cada una de las fases y las actividades que implica cada fase:

<detalle de cada fase>!!!1
