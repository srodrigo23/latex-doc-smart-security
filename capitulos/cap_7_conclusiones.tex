\chapter{Conclusiones y Recomendaciones}
\section{Conclusiones}
Las siguientes conclusiones se definen en base a los objetivos específicos planteados en la elaboración del proyecto.\\

\begin{itemize}
    \item La transmisión de datos por la red es el principal medio de comunicación entre sistemas; el desarrollo de diferentes protocolos de red, basados en una arquitectura, ayudaron a proveer usos específicos a las redes de computadoras tanto locales como de Internet. Aplicar los conceptos básicos de la comunicación de sistemas permitio el envio de fotogramas entre los módulos del sistema planteado.
    
    \item La inteligencia artificial y sus diversas ramas proveen de herramientas innovadoras en el campo de la visión por computadora, permitiendo la automatización de tareas de video-vigilancia e industrias, y en general donde es necesaria la supervisión humana.
    
    \item Las redes neuronales son un modelo de aprendizaje artificial que ofrecen identificar caracteristicas a partir de un conjunto de datos generales, de esta manera, permitir la clasificación de la información, facilitando el proceso de automatización. El resultado del entretenimiento de la red neuronal fue útil para el clasificador usado en la detección de fuego.
    
    \item La transmisión de video es una de las aplicaciones de la transmisión de datos por la red, que permiten compartir contenido multimedia para diversos usos tanto para el entretenimiento como para usos industriales y de seguridad. Esta tecnología y el protocolo Http Live Streaming, permitió realizar la transmisión de video en vivo de lo que las cámaras estan capturando.
    
    \item Para el análisis de imágenes es necesario obtener el archivo con el formato correcto; debido a la transmisión de datos por la red, el formato no persiste por lo que es necesario realizar tareas adicionales para poder reconstruir la información recibida y realizar su análisis.
    
    \item El sistema de video-vigilancia inteligente permite automatizar la alerta inmediata ante situaciones de peligro por medio de la notificación por correo electrónico y la visualización de video en vivo.
\end{itemize}

\section{Recomendaciones}
Para finalizar, se recomienda que para escalar el presente proyecto se realize:
\begin{itemize}
    \item Distribuir las tareas de detección en diversos computadores distribuidos en red, para alivianar la carga de los detectores.
    \item Independizar el módulo de transmisión de video en vivo, debido a que es una carga considerable codificar y decodificar video.
    \item Proveer un almacen de imágenes que representan las capturas de cada uno de los detectores.
    \item Implementar una base de datos para guardar un histórico de las detecciónes realizadas.
    \item Implementar una aplicación web para centralizar toda la información según cuentas de usuario. 
\end{itemize}
