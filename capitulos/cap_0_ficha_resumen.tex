\chapter{\centering Ficha Resúmen}

\begin{center}
    \begin{minipage}{14cm}

    El presente proyecto tiene como objetivo implementar un sistema de video-vigilancia inteligente para la alerta inmediata ante situaciones de peligro en el hogar. Se define como situaciones de peligro, aquellas que ponen en riesgo la integridad física de la familia o integridad material del hogar. Se enfocaron los esfuerzos, únicamente a la detección de fuego, movimiento e intrusos. El medio elegido para la notificación es el correo electrónico con el apoyo de transmisión de video en vivo para una visualización inmediata.\\

    Se hace uso de la metodología de desarrollo ``Cascada'' debido a que es la que mejor se adapta al desarrollo de este tipo de sistemas, el cual en un inicio define los requerimientos y alcances. El sistema se subdivide en dos módulos: el módulos de cámaras y el módulo de servidor.  El módulo de cámaras es el encargado de manejar la conexión de una cámara web o una picamera (cámara de RaspberryPi) y capturar los fotogramas para ser enviados al módulo de servidor. El módulo de servidor realiza diversas tareas como ser: gestionar las conexiones del módulos de cámaras, recibir fotogramas de cada instancia del módulo de cámaras, realizar el análisis de imágenes por parte de cada detector (fuego, intrusos, movimiento), notificar al usuario por medio de correo electrónico, adjuntando capturas de la detección y un enlace de visualización de video en vivo, y por vía whatsapp adjuntando solamente capturas. Se aplicaron algoritmos de visión por computadora para los detectores planteados.\\

    Para probar la funcionalidad y verificar los resultados esperados se realizaron pruebas de aceptación de cada uno de los detectores, con su respectivo flujo de detección y notificación en cada caso de prueba. Las pruebas se realizaron en un ambiente local, con la implementación del módulo de cámaras y el módulo de servidor.\\
    \end{minipage}
\end{center}